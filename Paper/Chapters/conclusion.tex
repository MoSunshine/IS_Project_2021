\section{Conclusion}

As an overall result, we see that a hybrid energy production model is more profitable than offering both energy resources separately. We also see that the hybrid producer's advantage increases the more risk-averse the producer is. This may be caused by the fact that a producer can offer more power for a given day-ahead price than a single energy producer. This follows since the mean of the hybrid producer's energy production sums up while the variance does not. However, the variance is smaller than the sum of both separately. The anti-correlation of both energy resources causes this effect. 
Even if the case study only focuses on a single period and not a time period over a whole year, it may also be a positive effect that the whole year power production faces smaller variance due to the even stronger anti-correlation of solar irradiance and wind power over the year. 

While the case study is based on a time series analysis on historical data for sun and wind energy production, the prices and ratios for day-ahead, adjustment, and balancing prices are taken from the example presented in Chapter 6 of \cite{Conejo10} . It may be worth to fully analyse a more realistic scenario setting and conduct case study over an entire year as we only focused on time period of one day. Another drawback of this survey is that the model faces some simplifications and assumptions. It may be interesting to extend the model to achieve a more realistic environment. 