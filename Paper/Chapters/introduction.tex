\section{Introduction}

\laura{Nur mal grob runtergeschrieben}
\david{Fand die Einleitung schon sehr gut, habe versucht noch reinzubringen warum wir uns genau das setting anschauen und was wir anders als zum Kapitel machen wollen.}
Renewable energies are currently advancing and gaining an increasing share of the energy production. On the one hand, this is induced by the desire to decrease the carbon footprint and on the other hand, the population is facing a decreasing availability of fossil fuels like natural gas, coal and oil. The advantage of renewable energies is that apart from not requiring fuel, which induces zero fuel costs, it is emission-free and therefore supported by the government. However, these energy sources are also non-dispatchable and have the major disadvantage of uncertainty. Conejo et al. \cite{Conejo10} considered the case of a wind power producer. Due to the uncertainty, the wind power producer must rely on the energy traded on the balancing market. \changedByDK{Therefore, he} solved an optimisation problem \changedByDK{\sout{with the objective}} to maximise the expected profits from trading on the day-ahead market and the adjustment market while also minimising the costs incurred in the balancing market caused by energy deviations. 

\changedByDK{ In addition to the uncertainty, the wind producer also faces a substantial fluctuation in production throughout the year, caused by the strong dependence of wind availability over different months in a year. Considering a wind power producer in Europe, wind availability is much higher in the winter than summer months, see, e.g. \cite{W11}. }

For this case study, we consider the case of an energy producer producing both wind power as well as solar energy. Several studies have been carried out to see how both \changedByDK{resources' production variability}  can be decreased by exploiting the anti-correlation of wind speed and solar irradiance. Coker et al. \cite{Coker2013} considered a region in south-west Britain to assess the variability of wind, solar and tidal current energy resources. Santos-Alamillos et al. \cite{Santos-Alamillos} aimed at finding the optimal spatial distribution of wind and solar farms across the Southern Iberian Peninsula \sout{in an attempt} to minimise the resulting net variability. Bett et al. \cite{BETT16} analysed daily data for \sout{the area of} Great Britain and \changedByDK{found}  evidence for an overall anticorrelation between wind speed and solar irradiance. As a side product, they also discovered that solar variability is significantly higher than wind variability and that both variabilities are higher in winter than in summer. Inspired by these results, we wish to set up a pool trading model for an energy producer offering energy produced from solar and wind power plants.  
\changedByDK{This setting of a hybrid wind-solar power producer
	overcomes the strong fluctuation in production throughout the year due to the strong over year anti-correlation of wind speed and solar irradiance, see figure \ref{fig:overyear}. In addition to
	the anti-correlation between wind speed and solar irradiance over the whole year, there is also an anti-correlation between both over the day, even if it is much smaller. We will exploit this anti-correlation over the day and include it into the forecast
	for wind speed and solar irradiance \laura{Includen wir das nicht eher in den production quantity forecast?}. Therefore, we will apply the forecast for wind speed for a time $t$ in dependence on the previous times' wind speeds and previous times' solar irradiance to improve the forecast possibly. }

\begin{figure}[h!]
	\centering
	
	\begin{minipage}{0.8\textwidth}
		\subfloat[]{
			\centering
			\scalebox{0.4}{\includegraphics{Figures/year.jpg}}
		}
		\hfill
		\subfloat[]{
			\centering
			\scalebox{0.4}{\includegraphics{Figures/year2.png}}
		}
		
		\caption{The average annual profiles of solar irradiacne and wind speed \cite{W11}}\label{fig:overyear}
	\end{minipage}	
\end{figure}