\section{Introduction}

\laura{Nur mal grob runtergeschrieben}
Renewable energies are currently advancing and gaining an increasing share of the energy production. On the one hand, this is induced by the desire to decrease the carbon footprint and on the other hand, population is facing a decreasing availability of fossil fuels like natural gas, coal and oil. The advantage of renewable energies is that apart from not requiring fuel, which induces zero fuel costs, it is also emission-free and therefore supported by the government. However, these energy sources are also non-dispatchable and have the major disadvantage of uncertainty. Conejo et. al considered the case of a wind power producer. Due to the uncertainty, the wind power producer has to rely on the energy traded on the balancing market. He therefore solved an optimisation problem with the objective to maximise the expected profits from trading on the day-ahead market and the adjustment market while also minimising the costs incurred in the balancing market caused by energy deviations. 
For this case study, we consider the case of an energy producer producing both wind power as well as solar energy. Several studies have been carried out to see how the production variability of both resources can be decreased by exploiting the anticorrelation of wind speed and solar irradiance. Coker et. al. considered a south-west Britain to assess the variability of wind, solar and tidal current energy resources. Santos-Alamillos et. al. aimed at finding the optimal spatial distribution of wind and solar farms across the Southern Iberian Peninsula in an attempt to minimise the resulting net variability. Bett et. al. analysed daily data for the area of Great Britain and were able to find evidence for an overall anticorrelation between wind speed and solar irradiance. As a side product, they also discovered that solar variability is significantly higher than wind variability and that both variabilities are higher in winter than in summer. Inspired by these results, we wish to set up a pool trading model for an energy producer offering energy produced from solar as well as wind power plants.  
